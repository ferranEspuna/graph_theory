\documentclass{amsart}
\renewcommand{\baselinestretch}{1.1}
\setlength{\parskip}{2mm}

\pagestyle{empty}


\usepackage{amsthm}
\usepackage{amsmath}
\usepackage{amsfonts}
\usepackage{amssymb}
\usepackage{textcomp}
\usepackage{graphicx}
\usepackage[margin=70pt]{geometry}
\setlength{\parindent}{0pt}
\usepackage{mathtools}
\usepackage{enumitem}

\usepackage{xcolor}



\newtheorem*{theirtheorem}{Theorem}
\newtheorem*{theirproposition}{Proposition}


\theoremstyle{plain}
\newtheorem*{theorem}{\textbf{Theorem}}
\newtheorem*{lemma}{\textbf{Lemma}}
\newtheorem*{corollary}{\textbf{Corollary}}
\newtheorem*{proposition}{\textbf{Proposition}}
\newtheorem*{claim}{\textbf{Claim}}
\newtheorem*{conjecture}{\textbf{Conjecture}}

\theoremstyle{definition}
\newtheorem*{rk}{\textbf{Remark}}

\newcommand{\Summ}[1]{\underset{#1}{\sum}}
\newcommand{\sti}[2]{\left\{\begin{array}{c} #1 \\ #2 \end{array}\right\}}

\newcommand{\diam}{\emph{diam}}
\newcommand{\conv}{\mbox{Conv}}
\newcommand{\C}{\mathcal {C}}
\newcommand{\R}{\mathbb{R}}
\newcommand{\Z}{\mathbb{Z}}
\newcommand{\N}{\mathbb{N}}
\newcommand{\F}{\mathbb{F}}

\newcommand{\B}{\mathcal{B}}
\newcommand{\A}{\mathcal{A}}
\newcommand{\G}{\mathcal{G}}
\newcommand{\D}{\mathcal{D}}

\DeclarePairedDelimiter{\norm}{\lVert}{\rVert}

\newcommand{\ov}[1]{\overline{#1}}

\newcommand{\nn}{\nonumber}

\newcommand{\st}{2}

\thispagestyle{empty}

\begin{document}

    {\Large Graph Theory -- MAMME}
    {\Large Session 12 -- Extremal Graph Theory II: Regularity Lemma}

    \vspace{0.5cm}

    \hrule

    \vspace{0.5cm}

    \noindent \textbf{Problem 4:}
    Prove the following color version of Szemerédi's Regularity Lemma:
    For every $\varepsilon > 0$ and $r \in \N$ there exists $M \in \N$ such that
    for every edge-colored graph $G$ with $r$ colors, there exists a partition of the vertex set
    $V(G) = V_1 \cup \ldots \cup V_k$ with $M \leq k \leq M$ such that
    all but at most $\varepsilon k^2$ pairs $(V_i, V_j)$ are $\varepsilon$-regular in each color class.


    \paragraph{\textbf{Solution (by Ferran Espuña):}}

    The notion of $\varepsilon$-regularity stated in the problem is slightly different from the one we saw in class.
    We will for now ignore this and address it later.
    To start, we will prove:

    \begin{proposition}
        For every $\varepsilon > 0$ and $r \in \N$ there exists $M \in \N$ such that
        for every graph $G$ on $n$ vertices, edge-colored with $r$ colors, there exists a partition of the vertex set
        $V(G) = V_1 \cup \ldots \cup V_k$ with $M \leq k \leq M$ such that, in each color class,

        \[
             \sum_{\substack{V_i, V_j \text{ not } \\ \varepsilon-regular}} \lvert V_i \rvert \lvert V_j \rvert \leq \varepsilon n^2
        \]

        That is, the partition is $\varepsilon$-regular in each color class.

        \begin{proof}
            The proof is analogous to the Szemerédi's Regularity Lemma proof we saw in class.
            We start with the trivial partition and keep refining it.
            For a partition $\pi$ and a color $c$, let

            \[
                d_2(\pi, c) = \sum_{V_i, V_j \in \pi} \frac{\lvert V_i \rvert \lvert V_j \rvert}{n^2} d_c(V_i, V_j)^2
            \]

            where the density $d_c(V_i, V_j)$ is the proportion of pairs $(v_i, v_j)$ that are colored $c$.
            We know from class that this quantity does not decrease when refining the partition.
            Also, we know that if $\pi$ is not $\varepsilon$-regular, in color class $c$, then there exists
            a refinement $\pi'$ such that $\lvert \pi' \rvert \leq \lvert \pi \rvert 2^{2 \lvert \pi \rvert}$ and

            \[
                d_2(\pi', c) \geq d_2(\pi, c) + \varepsilon^5
            \]

            Finally, we also know that this quantity is between $0$ and $1$.
            We can apply the same density increment argument as in class to the vector

            \[
                \left( d_2(\pi, 1), \ldots, d_2(\pi, r) \right)
            \]

            In each step, if there is still a color class $i$ such that $\pi$ is not $\varepsilon$-regular,
            in the color class $i$, we can refine the partition to increase the $i$th coordinate of the vector
            by at least $\varepsilon^5$, while the other coordinates don't decrease.
            Therefore, this process
            must stop after at most $\frac{r}{\varepsilon^5}$ steps,
            at which point the partition is $\varepsilon$-regular in each color class
            and the number of parts is bounded by a tower of $2$s of height at most $\frac{2r}{\varepsilon^5}$. \qedhere

        \end{proof}
    \end{proposition}

    \newpage

    The question now is how we can adapt this proof to the different notion of
    $\varepsilon$-regularity stated in the problem.
    The two would be equivalent if it were not for the fact that the sizes of the parts of a partition can
    be very different.
    We need to find a way to refine a partition at the end of each step
    such that the sizes of the parts at the end
    are more balanced.
    Given a partition with $k$ parts, we will aim for about $k^2$ parts.
    Suppose that

    \[
        (a - 1) \frac{n}{k^2} < \lvert V_i \rvert \leq a \frac{n}{k^2}
    \]

    Then, we will split $V_i$ into $a$ parts of size at most $\frac{n}{k^2}$.
    This will give us a partition with number of parts $k'$ satisfying

    \[
        k^2 = \sum_{i} \frac{\lvert V_i \rvert k^2}{n} \leq k' \leq \sum_{i} 1 + \frac{\lvert V_i \rvert k^2}{n} = k^2 + k
    \]

    Note that the upper bound is just a function of $k$.
    This means that the argument above works exactly the same but with a higher
    bound on the number of parts.
    However, at each step, we may assume that any part $V_i'$ satisfies

    \begin{equation} \label{eq:bound}
        k' \lvert V_i' \rvert \leq \frac{n}{k^2}(k^2 + k) = n\left(1 + \frac{1}{k}\right) \leq  n\left(1 + \hat{\varepsilon} \right)
    \end{equation}

    Where we can make $\hat{\varepsilon} \coloneqq \frac{1}{k}$ as small as we'd like depending on $\varepsilon$
    by starting the process with an arbitrary partition with a large number of parts.

    Now, suppose that, for some color class $c$, we have more than
    $\varepsilon k^2$ pairs $(V_i, V_j)$ that are not $\varepsilon$-regular.
    We can write this as

    \begin{equation} \label{eq:many_pairs}
        \sum_{\substack{V_i, V_j \text{ not } \\ \varepsilon-regular}} 1 \geq \varepsilon k^2
    \end{equation}

    We will do the same density increment argument as before, but now with $\varepsilon' = \frac{\varepsilon}{2}$.
    Suppose, by way of contradiction, that

    \begin{equation}\label{eq:contradiction}
        \sum_{\substack{V_i, V_j \text{ not } \\ \varepsilon'-regular}} \lvert V_i \rvert \lvert V_j \rvert < \varepsilon' n^2 = \frac{\varepsilon n^2}{2}
    \end{equation}

    Multiplying~\eqref{eq:many_pairs} by $\left(\frac{n}{k}\right)^2$ and substracing~\eqref{eq:contradiction} from it, we get

    \begin{equation}\label{eq:combined}
        \sum_{\substack{V_i, V_j \text{ not } \\ \varepsilon'-regular}} \left( \frac{n}{k} \right)^2 - \lvert V_i \rvert \lvert V_j \rvert > \frac{\varepsilon n^2}{2}
    \end{equation}

    Note, however, that if we extend the sum to all pairs $(V_i, V_j)$, (including the same part twice), we get $-\text{Var}(X) < 0$,
    where $X$ is the size of a part of the partition chosen uniformly at random.
    This means that there is a set of pairs $I \subset \pi^2$ satisfying

    \begin{equation*}\label{eq:others}
        \sum_{(V_i, V_j) \in I} \left( \frac{n}{k} \right)^2 - \lvert V_i \rvert \lvert V_j \rvert <  - \frac{\varepsilon n^2}{2}
        \implies \sum_{(V_i, V_j) \in I} \lvert V_i \rvert \lvert V_j \rvert > \frac{\varepsilon n^2}{2} + \lvert I \rvert \left( \frac{n}{k} \right)^2
    \end{equation*}

    Multiplying by $k^2$, and applying the bound on the relative size of the parts we obtained in~\eqref{eq:bound}, we get

    \begin{equation*}\label{eq:equation}
        n^2 \left( \frac{\varepsilon k^2}{2} + \lvert I \rvert \right) < \lvert I \rvert \, n^2 \left( 1 + \hat{\varepsilon} \right)
        \implies \frac{\varepsilon k^2}{2 \lvert I \rvert} + 1 < 1 + \hat{\varepsilon}
        \implies \frac{\varepsilon k^2}{2 \lvert I \rvert} < \hat{\varepsilon}
    \end{equation*}

    But obviously $\lvert I \rvert \leq k^2$ so $\frac{\varepsilon}{2} < \hat{\varepsilon}$, which is a contradiction
    if we choose $\varepsilon$ small enough.
    This means that~\eqref{eq:contradiction} cannot hold,
    so the partition is not $\varepsilon'$-regular in color class $c$, allowing us to refine
    the partition further.
    The rest of the argument follows as before, but the density increment is now $(\frac{\varepsilon}{2})^5$
    and $\lvert \pi' \rvert \leq f\left( \lvert \pi \rvert 2^{2 \lvert \pi \rvert}\right)$,
    where $f(k) = k(k+1)$.

\end{document}

