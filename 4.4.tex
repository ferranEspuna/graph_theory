\documentclass{amsart}
\renewcommand{\baselinestretch}{1.1}
\setlength{\parskip}{2mm}

\usepackage{amsthm}
\usepackage{amsmath}
\usepackage{amsfonts}
\usepackage{amssymb}
\usepackage{textcomp}
\usepackage{graphicx}
\usepackage[margin=70pt]{geometry}
\usepackage{mathtools}

\newtheorem*{theirtheorem}{Theorem}
\newtheorem*{theirproposition}{Proposition}


\theoremstyle{plain}
\newtheorem*{theorem}{\textbf{Theorem}}
\newtheorem*{lemma}{\textbf{Lemma}}
\newtheorem*{corollary}{\textbf{Corollary}}
\newtheorem*{proposition}{\textbf{Proposition}}
\newtheorem*{claim}{\textbf{Claim}}
\newtheorem*{conjecture}{\textbf{Conjecture}}

\theoremstyle{definition}
\newtheorem*{rk}{\textbf{Remark}}

\newcommand{\Summ}[1]{\underset{#1}{\sum}}
\newcommand{\sti}[2]{\left\{\begin{array}{c} #1 \\ #2 \end{array}\right\}}

\newcommand{\diam}{\emph{diam}}
\newcommand{\conv}{\mbox{Conv}}
\newcommand{\C}{\mathcal {C}}
\newcommand{\R}{\mathbb{R}}
\newcommand{\Z}{\mathbb{Z}}
\newcommand{\N}{\mathbb{N}}
\newcommand{\F}{\mathbb{F}}

\newcommand{\B}{\mathcal{B}}
\newcommand{\A}{\mathcal{A}}
\newcommand{\G}{\mathcal{G}}
\newcommand{\D}{\mathcal{D}}

\newcommand{\ov}[1]{\overline{#1}}

\newcommand{\nn}{\nonumber}

\def\st{2}

\thispagestyle{empty}

\begin{document}

    {\Large Graph Theory -- MAMME}
    {\Large Chapter 5 -- Random Graphs III}

    \vspace{0.5cm}

    \hrule

    \vspace{0.5cm}

    \noindent \textbf{Problem 4:}
    Prove Lemma 4.1. from the notes:


    \paragraph{\textbf{Solution (by Ferran Espuña):}}
    Let us recall the statement of the lemma:
    Let $\mathbb{G}_{n,p(n)}$ be the random graph where $p(n) = \frac{c}{n}$ and $c > 1$ is a constant.
    Let us fix a vertex $v$ and start a BFS from $v$, exploring the graph in the order
    $x_1 = v,\, x_2 \in \mathcal{N}(v), \cdots$.
    Let $A_k$ be the set of vertices in the BFS queue after $k$ steps (that is, those that have appeared
    as neighbors of some vertex $x_t$ for $t \leq k$ but are not themselves of the form $x_t$ for some $t \leq k$).
    For example,$
    \, A_0 = \{ v = x_1 \}, \,
    A_1 = \mathcal{N}(x_1) \setminus \{x_1\} = \mathcal{N}(x_1), \,
    A_2 = (\mathcal{N}(x_1) \cup \mathcal{N}(x_2)) \setminus \{x_1, x_2\}, \, \cdots$

    \begin{lemma}
        With this notation, $a.a.s.$, either:
        \begin{enumerate}
            \item the process stops before $k^-$ steps, or \label{itm:1}
            \item $|A_k| \geq \frac{c}{2}k$ for all $k \in [k^{-}, \, k^{+}]$
            and thus the process survives until $k^{+}$. \label{itm:2}
        \end{enumerate}
        Where $k^{-} = \log(n)$ and $k^{+} = n^{\frac{2}{3}}$.
    \end{lemma}

    \begin{proof}
        \vspace{-3mm}
        We will assume condition~\ref{itm:1} does not hold, and we will bound the probability of
        condition~\ref{itm:2} \emph{failing for the first time} at some $k \in [k^{-}, \, k^{+}]$.
        For this, note that condition~\ref{itm:2} holding for all previous steps implies that
        the process has not stopped before $k$.
        Furthermore, there is a previous step $\hat{k}$ for which the set of vertices explored in the BFS
        between steps $\hat{k}$ and $k$ is already predetermined at time $\hat{k}$,
        because they are all in $A_{\hat{k}}$.
        indeed, we know for all $k' \in [k^{-}, \, k]$ that $|A_{k'}| \geq \frac{c}{2}k'$.
        If we set $\frac{c}{2}\hat{k} \geq k - \hat{k}$, what we have just said is true.
        A valid choice for $\hat{k}$ is $\lceil \frac{2}{3} k \rceil \geq \frac{c+2}{2}k$.
        A different argument is needed for  $\lceil \frac{2}{3} k \rceil < k^-$, but we
        will worry about that later.

        % TODO this does NOT work, the 3 in the denominator fucks everything up
        % Try a different approach, setting Y_k to be all nodes \emph{seen} up to step k
        \noindent We will assume a model in which the edges of the graph are created at
        random when the BFS algorithm needs them.
        That is, at step $t$, the edges from $x_t$ are determined.
        The probability $p_{u, k}$ of a given vertex $u \notin \{x_1, \cdots x_{k}\}$
        being in $A_k$ bounded below by so of the probability $p_{u, k; \hat{k}}$
        of it entering the BFS queue between steps $\hat{k}$ and $k$.
        This probability can be calculated right after step $\hat{k}$,
        and by symmetry is independent of everything explored previously.
        Because there are $k - \hat{k} \geq \frac{k+3}{3}$ chances for it to happen between $\hat{k}$ and $k$,
        we have that
        \begin{equation}
            p_{u, k; \hat{k}} \geq 1 - (1 - p)^{\frac{k+3}{3}}
            \label{eq:equation1}
        \end{equation}
        And because there are $n - k$ vertices not in $\{x_1, \cdots x_k\}$, we have that
        \begin{equation}
            |A_k| \geq \text{Binom}\left(n - k,\, 1 - (1 - p)^{\frac{k+3}{3}}\right)
            \label{eq:equation2}
        \end{equation}
        with expectation
        \begin{equation}
            \mathbb{E}(|A_k|) \geq (n - k)\left(1 - (1 - p)^{\frac{k+3}{3}}\right)
            \label{eq:equation_expec}
        \end{equation}
        To apply Chernoff's bound, we need to bound
        \begin{equation}
            f(k) \coloneqq \frac{\frac{c-1}{2}k}{(n - k)\left(1 - (1 - p)^{\frac{k+3}{3}}\right)}
            \label{eq:f}
        \end{equation}
        As it turns out, $f(k)$ is increasing in $k$. % todo: prove this
        Plugging in $k = k^{+}$, we get, as $n \to \infty$,
        \begin{equation}
            f(k) \leq \frac
            {(c-1)n^{\frac{2}{3}}}
            {2(n-n^{\frac{2}{3}})\left(1 - (1 - \frac{c}{n})^{\frac{n^{2/3}}{4}}\right)}
            \leq \frac{c-1}
            {n^{\frac{1}{3}}\left(1 - e^{-\frac{cn^{-1/3}}{4}}\right)}
            \label{eq:fbound}
        \end{equation}



    \end{proof}







\end{document}

