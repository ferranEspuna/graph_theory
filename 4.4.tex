\documentclass{amsart}
\renewcommand{\baselinestretch}{1.1}
\setlength{\parskip}{2mm}

\pagestyle{empty}


\usepackage{amsthm}
\usepackage{amsmath}
\usepackage{amsfonts}
\usepackage{amssymb}
\usepackage{textcomp}
\usepackage{graphicx}
\usepackage[margin=70pt]{geometry}
\usepackage{mathtools}

\newtheorem*{theirtheorem}{Theorem}
\newtheorem*{theirproposition}{Proposition}


\theoremstyle{plain}
\newtheorem*{theorem}{\textbf{Theorem}}
\newtheorem*{lemma}{\textbf{Lemma}}
\newtheorem*{corollary}{\textbf{Corollary}}
\newtheorem*{proposition}{\textbf{Proposition}}
\newtheorem*{claim}{\textbf{Claim}}
\newtheorem*{conjecture}{\textbf{Conjecture}}

\theoremstyle{definition}
\newtheorem*{rk}{\textbf{Remark}}

\newcommand{\Summ}[1]{\underset{#1}{\sum}}
\newcommand{\sti}[2]{\left\{\begin{array}{c} #1 \\ #2 \end{array}\right\}}

\newcommand{\diam}{\emph{diam}}
\newcommand{\conv}{\mbox{Conv}}
\newcommand{\C}{\mathcal {C}}
\newcommand{\R}{\mathbb{R}}
\newcommand{\Z}{\mathbb{Z}}
\newcommand{\N}{\mathbb{N}}
\newcommand{\F}{\mathbb{F}}

\newcommand{\B}{\mathcal{B}}
\newcommand{\A}{\mathcal{A}}
\newcommand{\G}{\mathcal{G}}
\newcommand{\D}{\mathcal{D}}

\newcommand{\ov}[1]{\overline{#1}}

\newcommand{\nn}{\nonumber}

\def\st{2}

\thispagestyle{empty}

\begin{document}

    {\Large Graph Theory -- MAMME}
    {\Large Chapter 5 -- Random Graphs III}

    \vspace{0.5cm}

    \hrule

    \vspace{0.5cm}

    \noindent \textbf{Problem 4:}
    Prove Lemma 4.1. from the notes:


    \paragraph{\textbf{Solution (by Ferran Espuña):}}
    Let us recall the statement of the lemma:
    Let $\mathbb{G}_{n,p(n)}$ be the random graph where $p(n) = \frac{c}{n}$ and $c > 1$ is a constant.
    Let us fix a vertex $v$ and start a BFS from $v$, exploring the graph in the order
    $x_1 = v,\, x_2 \in \mathcal{N}(v), \cdots$.
    Let $A_k$ be the set of vertices in the BFS queue after $k$ steps (that is, those that have appeared
    as neighbors of some vertex $x_t$ for $t \leq k$ but are not themselves of the form $x_t$ for some $t \leq k$).
    For example,$
    \, A_0 = \{ v = x_1 \}, \,
    A_1 = \mathcal{N}(x_1) \setminus \{x_1\} = \mathcal{N}(x_1), \,
    A_2 = (\mathcal{N}(x_1) \cup \mathcal{N}(x_2)) \setminus \{x_1, x_2\}, \, \cdots$

    \begin{lemma}
        With this notation, $a.a.s.$, either:
        \begin{enumerate}
            \item the process stops before $k^-$ steps, or \label{itm:1}
            \item $|A_k| \geq \frac{c - 1}{2}k$ for all $k \in [k^{-}, \, k^{+}]$
            and thus the process survives until $k^{+}$. \label{itm:2}
        \end{enumerate}
        Where $k^{-} = M(c)\log(n)$ and $k^{+} = n^{\frac{2}{3}}$.
    \end{lemma}

    \begin{proof}
        \vspace{-3mm}
        We will assume condition~\eqref{itm:1} does not hold, and we will bound the probability of
        condition~\eqref{itm:2} \emph{failing for the first time} at some $k \in [k^{-}, \, k^{+}]$.
        For this, note that condition~\eqref{itm:2} holding for all previous steps implies that
        the process has not stopped before $k$.

        \noindent Let $Z_k$ be the set of vertices seen up to step $k$.
        That is, $Z_k = A_k \cup \{x_1, \cdots, x_k\}$ so that $|Z_k| = |A_k| + k$.
        We will shift our focus to $Z_k$ and bound the probability of $|Z_k| < \frac{ck}{2} < \frac{c-1}{2}k + k$.
        Because we have only conditioned on the total number of seen vertices being high,
        the number of vertices in $Z_k$ distributed at least as a binomial
        where each vertex has $k$ independent chances of being seen (once at each step $k'$ up to $k$).
        That is,
        \begin{equation}
            |Z_k| \geq \overline{Z_k} \sim \text{Binom}\left(n, 1 - (1 - p)^k\right)
            \label{eq:binom}
        \end{equation}
        with expectation
        \begin{equation}
            \mathbb{E}\left(\overline{Z_k}\right) = n\left(1 - (1 - p)^k\right)
            \label{eq:expec}
        \end{equation}
    To apply Chernoff's bound, we need to bound above
    \begin{equation}
        1 - \delta = f(k) \coloneqq \frac{ck/2}{n\left(1 - (1 - p)^k\right)} =
        \frac{ck}{2n\left(1 - (1 - c/n)^k\right)} \leq \frac{ck}{2n\left(1 - e^{-ck/n}\right)}
        \label{eq:ff}
    \end{equation}
    \begin{claim}
        This bound is increasing in $k$ % todo: prove this
    \end{claim}
    \noindent Plugging in $k = k^{+}$, we get, as $n \to \infty$,
    \begin{equation}
        f(k) \leq
        \frac{cn^{2/3}}{2n\left(1 - e^{-cn^{-1/3}}\right)} \leq
        \frac{cn^{-1/3}}{(2-\epsilon)\left(cn^{-1/3}\right)} = \frac{1}{2-\epsilon} < 1
        \label{eq:fbound_2}
    \end{equation}
    For some small $\epsilon > 0$.
    Therefore, the $\delta$ in Chernoff's bound is positive, and we can apply it:
    \begin{equation}
        \mathbb{P}\left(\overline{Z_k} < \frac{ck}{2}\right) =
        \mathbb{P}\left(\overline{Z_k} < (1 - \delta)\mathbb{E}(\overline{Z_k})\right) \leq
        e^{-\frac{\delta^2}{2}\mathbb{E}(\overline{Z_k})}
        \label{eq:chernoff}
    \end{equation}
    \noindent Finally, we will union bound over all $k \in [k^{-}, \, k^{+}]$.
    Because $k \geq k^-$, each term is at most
    \begin{equation}
        e^{-\frac{\delta^2}{2}\mathbb{E}(\overline{Z_{k^-}})} \leq
        e^{-\frac{\delta^2ck^-}{4}} \leq
        e^{-\frac{c\delta^2M(c)\log(n)}{4}} =
        n^{-\frac{cM(c)\delta^2}{4}}
        \label{eq:logbound}
    \end{equation}
    So the probability of the condition failing at any point in $[k^{-}, \, k^{+}]$ is at most
    \begin{equation}
        n^{\frac{2}{3}}n^{-\frac{cM(c)\delta^2}{4}} = n^{\frac{2}{3} - \frac{cM(c)\delta^2}{4}} =
        n^{\frac{8 - 3cM(c)\delta^2}{12}}
        \label{eq:finalbound}
    \end{equation}
    \noindent For this to be $o(1)$, we need $8 - 3cM(c)\delta^2$ to be negative.
    To give a sufficient condition, take $\epsilon = \frac{1}{2}$ Then $\delta =  1 - \frac{2}{3} = \frac{1}{3}$.
    We need $8 - \frac{cM(c)}{3} < 0$, which is equivalent to $M(c) > \frac{24}{c}$.
    \end{proof}

\end{document}

