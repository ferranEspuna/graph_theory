\documentclass{amsart}
\renewcommand{\baselinestretch}{1.1}
\setlength{\parskip}{2mm}

\pagestyle{empty}


\usepackage{amsthm}
\usepackage{amsmath}
\usepackage{amsfonts}
\usepackage{amssymb}
\usepackage{textcomp}
\usepackage{graphicx}
\usepackage[margin=70pt]{geometry}
\setlength{\parindent}{0pt}
\usepackage{mathtools}
\usepackage{enumitem}

\usepackage{xcolor}

\newtheorem*{theirtheorem}{Theorem}
\newtheorem*{theirproposition}{Proposition}


\theoremstyle{plain}
\newtheorem*{theorem}{\textbf{Theorem}}
\newtheorem*{lemma}{\textbf{Lemma}}
\newtheorem*{corollary}{\textbf{Corollary}}
\newtheorem*{proposition}{\textbf{Proposition}}
\newtheorem*{claim}{\textbf{Claim}}
\newtheorem*{conjecture}{\textbf{Conjecture}}

\theoremstyle{definition}
\newtheorem*{rk}{\textbf{Remark}}

\newcommand{\Summ}[1]{\underset{#1}{\sum}}
\newcommand{\sti}[2]{\left\{\begin{array}{c} #1 \\ #2 \end{array}\right\}}

\newcommand{\diam}{\emph{diam}}
\newcommand{\conv}{\mbox{Conv}}
\newcommand{\C}{\mathcal {C}}
\newcommand{\R}{\mathbb{R}}
\newcommand{\Z}{\mathbb{Z}}
\newcommand{\N}{\mathbb{N}}
\newcommand{\F}{\mathbb{F}}

\newcommand{\B}{\mathcal{B}}
\newcommand{\A}{\mathcal{A}}
\newcommand{\G}{\mathcal{G}}
\newcommand{\D}{\mathcal{D}}

\DeclarePairedDelimiter{\norm}{\lVert}{\rVert}

\newcommand{\ov}[1]{\overline{#1}}

\newcommand{\nn}{\nonumber}

\newcommand{\st}{2}

\thispagestyle{empty}

\begin{document}

    {\Large Graph Theory -- MAMME}
    {\Large Session 8 -- Spectral Graph Theory II}

    \vspace{0.5cm}

    \hrule

    \vspace{0.5cm}

    \noindent \textbf{Problem 4:}
    Let $T_{d, k}$ be a $(d-1)$-ary rooted tree of height $k$.
    Show that

    \begin{enumerate}[label=(\alph*)]
        \item $\lambda_1(T_{d, k}) \leq 2 \sqrt{d - 1}$ \label{item:a}
        \item $\lim_{k \to \infty} \lambda_1(T_{d, k}) = 2 \sqrt{d - 1}$ \label{item:b}
    \end{enumerate}


    \paragraph{\textbf{Solution (by Ferran Espuña):}}
    Let $A$ be the adjacency matrix of $T_{d, k}$.
    For example, if $d = 3$ and $k = 2$,
    \[A = \begin{pmatrix}
        0  & \color{red} 1  & \color{red} 1  &  0  &  0  &  0  &  0 \\
        \color{red} 1  &  0  &  0  & \color{red} 1  & \color{red} 1  &  0  &  0 \\
        \color{red} 1  &  0  &  0  &  0  &  0  & \color{red} 1  & \color{red} 1 \\
        0  & \color{red} 1  &  0  &  0  &  0  &  0  &  0 \\
        0  & \color{red} 1  &  0  &  0  &  0  &  0  &  0 \\
        0  &  0  & \color{red} 1  &  0  &  0  &  0  &  0 \\
        0  &  0  & \color{red} 1  &  0  &  0  &  0  &  0
    \end{pmatrix}
    \]

    where the first row corresponds to the root of the tree,
    the second and third rows correspond to its children and the other four rows correspond to the leaves.
    Now we can rephrase~\ref{item:a} as the following:

    \begin{proposition} \label{prop:a}
        $\norm{A}_2 \leq 2 \sqrt{d - 1}$.
        \begin{proof}
            We will use this famous inequality from linear algebra:
            $\norm{B}_2 \leq \sqrt{ \norm{B}_1 \cdot \norm{B}_\infty }$.
            In this case, we will apply this to powers of $A$, which is non-negative and symmetric.
            Therefore, for $B = A^p$

            \begin{equation*}
                \norm{B}_1 = \norm{B}_{\infty} = \max_{i} \sum_{j} B_{ij}
            \end{equation*}

            so in particular

            \begin{equation}
                \norm{A}_2^p = \norm{A^p}_2 \leq \sqrt{\norm{A^p}_1 \cdot \norm{A^p}_{\infty}} = \max_{i} \sum_{j} A^p_{ij}
                \label{eq:norm_bound}
            \end{equation}

            Which is just a row sum of $A^p$.
            This corresponds to the number $P_{i, p}$ of paths of length $p$ from a given vertex $v_i$.
            Let's try to approximate this number.

            At each step of the path, the height (distance from the root) either increases by $1$
            (we say we go \emph{down})
            or decreases by $1$ (we say we go \emph{up}).
            Because both at the start and end of the path the height must be in the range $[0, k]$,
            the difference between the number of \emph{up} steps $s_{u}$
            and the number of \emph{down} steps $s_{d}$ must satisfy

            \begin{equation*}
                \lvert s_{u} - s_{d} \rvert \leq k
            \end{equation*}
            and obviously
            \begin{equation*}
                s_{u} + s_{d} = p
            \end{equation*}
            All together,
            \begin{equation*}
                \frac{p-k}{2} \leq s_d \leq  \frac{p+k}{2}
            \end{equation*}

            If we go \emph{up}, there is only one choice (the parent of the current node).
            If we go \emph{down}, there are $d-1$ choices (the children of the current node).
            Furthermore, for each choice of $s_d$ there are $\binom{p}{s_d}$
            ways to choose the \emph{down} steps.
            \begin{rk}
                Not all the ways of choosing are valid, as we should also ensure that the
                height stays within the range $[0, k]$ during the whole path.
                However, because we are only interested in an upper bound,
                we are free to ignore this.
            \end{rk}

            All in all, for all $i$,
            \begin{equation*}
                P_{i,p}
                \leq
                \sum_{s_d = (p-k)/2}^{(p+k)/2} \binom{p}{s_d} (d-1)^{s_d}
                \leq
                k \cdot 2^p \cdot (d-1)^{\frac{p+k}{2}}
            \end{equation*}

            Putting this together with~\eqref{eq:norm_bound}, we get
            \begin{equation*}
                \norm{A}_2^p \cdot k \cdot 2^p \cdot (d-1)^{\frac{p}{2}} \implies
                \norm{A}_2 \leq \lim_{p \to \infty}
                \left(k \cdot 2^p \cdot (d-1)^{\frac{p+k}{2}}\right)^{\frac{1}{p}} = 2 \sqrt{d-1}
            \end{equation*}
        \end{proof}
    \end{proposition}

    Similarly, we can rephrase~\ref{item:b} as the following:
    \begin{proposition}
        $\lim_{k \to \infty} \norm{A(k)}_2 = 2 \sqrt{d - 1}$.

        \begin{proof}
            We would now like to bound \emph{below} the 2-norm of $A$, so will use the inequality
            $\norm{B}_2 \geq \frac{1}{\sqrt{n}} \norm{B}_\infty$, where $n$ is the size of the matrix.
            \begin{rk}
                In our case,
                \begin{equation*}
                    n = \sum_{i=0}^{k} (d-1)^i = \frac{(d-1)^{k+1} - 1}{d-2} < (d-1)^{k+1}
                \end{equation*}
            \end{rk}

            We repeat the same calculation for the number of paths of length $p$ from a given vertex $v_i$,
            but now we need to bound it below.
            Because of the maximum in the formula in the $\infty$-norm,
            we're free to choose the starting vertex $v_i$.
            We will choose the root.
            We're also free to choose $p = 2mk$ for some $m \in \N$.

            To make the count easier, we will consider only some of the paths.
            Specifically, we will consider paths that pass through the root
            at the $2kj$-th step for all $j \in [0, m]$.

            \begin{rk}
                The number of \emph{down} steps is $mk$
                because the paths start and end at the root so $s_u = s_d$.
            \end{rk}

            \begin{rk}
                The number of valid orderings for the \emph{up} and \emph{down}
                steps for the steps in between
                $2jm$ and $2(j+1)m$ is
                the corresponding \emph{Catalan number}:
                \begin{equation*}
                    C_{k} = \frac{1}{k+1} \binom{2k}{k}
                \end{equation*}
                This is because the path can go as far down as it wants,
                as the leaves are at distance $k$ from the root.
                Therefore, there are $C_k^m$ ways to order the
                \emph{up} and \emph{down} steps for the whole path.
            \end{rk}

            All together,
            \begin{align*}
                    \norm{A}_2^{2mk}
                    & = \norm{A^{2mk}}_2
                    \geq \frac{1}{\sqrt{n}} \norm{A^{2mk}}_\infty
                    \geq \frac{1}{\sqrt{n}} C_{k}^m (d-1)^{mk} =
                    \frac{(d-1)^{mk}}{(k+1)^m \cdot \sqrt{n}} \binom{2k}{k}^{m}  \geq \\
                    & \geq \frac{4^{mk} \cdot (d-1)^{mk}}{(k+1)^m \cdot \sqrt{n} \cdot 2^m \cdot k^{m/2}} >
                    \frac{4^{mk} \cdot (d-1)^{mk - (k+1)/2}}{(k+1)^m \cdot 2^m \cdot k^{m/2}}
            \end{align*}

            If we now set, for example, $m = k$, we get that

            \begin{equation*}
                \liminf_{k \to \infty} \norm{A(k)}_2 \geq
                \lim_{k \to \infty} \left( \frac{4^{mk} \cdot (d-1)^{mk - (k+1)/2}}{(k+1)^m \cdot 2^m \cdot k^{m/2}} \right)^{\frac{1}{2mk}} =
                \lim_{k \to \infty} 2 \frac{(d-1)^{1/2 \, - \, 1/4k\,  - \, 1/4k^2}}{(k+1)^{1/2k} \cdot 2^{1/2k} \cdot k^{1/4k}}
                = 2 \sqrt{d-1}
            \end{equation*}

            Together with part~\ref{item:a}, we have
             $\lim_{k \to \infty} \norm{A(k)}_2 = 2 \sqrt{d-1}$

        \end{proof}
    \end{proposition}




\end{document}

