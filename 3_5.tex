\documentclass{amsart}
\renewcommand{\baselinestretch}{1.1}
\setlength{\parskip}{2mm}

\usepackage{amsthm}
\usepackage{amsmath}
\usepackage{amsfonts}
\usepackage{amssymb}
\usepackage{fullpage}
\usepackage{xcolor}
\usepackage{textcomp}
\usepackage{graphicx}

\newtheorem*{theirtheorem}{Theorem}
\newtheorem*{theirproposition}{Proposition}


\theoremstyle{plain}
\newtheorem*{theorem}{\textbf{Theorem}}
\newtheorem*{lemma}{\textbf{Lemma}}
\newtheorem*{corollary}{\textbf{Corollary}}
\newtheorem*{proposition}{\textbf{Proposition}}
\newtheorem*{claim}{\textbf{Claim}}
\newtheorem*{conjecture}{\textbf{Conjecture}}

\theoremstyle{definition}
\newtheorem*{rk}{\textbf{Remark}}

\newcommand{\Summ}[1]{\underset{#1}{\sum}}
\newcommand{\sti}[2]{\left\{\begin{array}{c} #1 \\ #2 \end{array}\right\}}

\newcommand{\diam}{\emph{diam}}
\newcommand{\conv}{\mbox{Conv}}
\newcommand{\C}{\mathcal {C}}
\newcommand{\R}{\mathbb{R}}
\newcommand{\Z}{\mathbb{Z}}
\newcommand{\N}{\mathbb{N}}
\newcommand{\F}{\mathbb{F}}

\newcommand{\B}{\mathcal{B}}
\newcommand{\A}{\mathcal{A}}
\newcommand{\G}{\mathcal{G}}
\newcommand{\D}{\mathcal{D}}

\newcommand{\ov}[1]{\overline{#1}}

\newcommand{\nn}{\nonumber}

\def\st{2}

\thispagestyle{empty}

\begin{document}

    {\Large Graph Theory -- MAMME}
    {\Large Chapter 3 -- Random Graphs I}

    \vspace{0.5cm}

    \hrule

    \vspace{0.5cm}

    \noindent \textbf{Problem 5:}
    Show that if $p(n) = n^{\alpha}$ for $\alpha < -3/2$, then
    a.a.s. $\mathbb{G}_{n, p(n)}$ consists of independent edges.


    \paragraph{\textbf{Solution (by Ferran Espuña):}}
    Note that the condition stated in the problem is equivalent to
    all vertices having degree less than $2$.
    For our purposes, we only need to show that it is sufficient:
    Indeed, if two edges share a vertex, then that vertex has degree at least $2$.

    \begin{proposition}
        If $p(n) = n^{\alpha}$ for $\alpha < -3/2$, then a.a.s.\
        all vertices in $\mathbb{G}_{n, p(n)}$ have degree less than $2$.

        \begin{proof}
            \vspace{-3mm}
            Fixing $n$, Let $G = \mathbb{G}_{n, p(n)}$ and $X$
            be the number of vertices of degree at least $2$ in $G$.
            Then,
            \begin{equation}
                \mathbb{E}(X) = \sum_{v\in(G)} \mathbb{P}(d(v) \geq 2)\label{eq:equation}
            \end{equation}
            However, for any $v \in V(G)$, we have that
            \begin{equation}
                d(v) \geq 2 \iff v \sim s \text{ and } v \sim t \text{ for some }
                s, t \in V(G) \text{ with } s \neq t; \, s, t \neq v
                \label{eq:equation2}
            \end{equation}
            Note that this condition does not depend on the order of $s$ and $t$.
            By the union bound, we have that
            \begin{equation}
                \mathbb{P}(d(v) \geq 2) \leq
                \sum_{(s, t) \in \binom{V(G) \setminus \{v\}}{2}} \mathbb{P}(v \sim s \text{ and } v \sim t) =
                \binom{n-1}{2} p(n)^2 < n^2 p(n)^2
                \label{eq:equation3}
            \end{equation}
            Substituting~\eqref{eq:equation3} into~\eqref{eq:equation}, we get
            \begin{equation}
                \mathbb{E}(X) < n \cdot n^2 p(n)^2 = n^3 p(n)^2 = n^{3 + 2\alpha}
                \label{eq:equation4}
            \end{equation}
            But since $\alpha < -3/2$, we have that $3 + 2\alpha < 0$ and thus $\mathbb{E}(X) \to 0$ as $n \to \infty$.
            By Markov's inequality, we have that
            \begin{equation}
                \mathbb{P}(X \geq 1) \leq \mathbb{E}(X) \to 0 \text{ as } n \to \infty
                \label{eq:equation5}
            \end{equation}
            And thus, a.a.s. $X = 0$, that is, all vertices have degree less than $2$.
        \end{proof}
    \end{proposition}






\end{document}

